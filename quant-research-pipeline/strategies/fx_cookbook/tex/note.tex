\documentclass[11pt,a4paper]{article}

\usepackage{amsmath,amssymb}
\usepackage{booktabs}
\usepackage{hyperref}
\usepackage[margin=1in]{geometry}
\usepackage{array}
\usepackage{longtable}

\hypersetup{colorlinks=true, linkcolor=blue, citecolor=blue, urlcolor=blue}

\title{Pre-Registered Research Note:\\
  Systematic FX Momentum and Carry Strategies}
\author{StratPipe — Quant Research Pipeline\\
  \texttt{strategy\_id: fx\_cookbook, version: v1.0}}
\date{February 2026}

\begin{document}
\maketitle

%% ============================================================
\begin{abstract}
We implement a systematic FX strategy combining multi-horizon momentum
and carry signals across 24 USD/FX pairs, following the framework of
Anand et al.~\cite{anand2019fxcookbook}. Momentum aggregates trend signs
over 21--252 day lookbacks with hysteresis noise control and dispersion
deflation. Carry normalises the forward--spot spread by return volatility.
Time-series portfolios use inverse volatility weighting; cross-sectional
portfolios are beta-neutralised to USD~PC1. This note pre-registers
hypotheses, data splits (train 2000--2014, validate 2015--2018, holdout
2019--2024 untouched), and success criteria before holdout evaluation.
The primary test is a two-sided $t$-test on annualised Sharpe ratios
($H_1$: Sharpe $> 0.4$ momentum, $> 0.5$ carry; $\alpha = 0.05$).
All 18 equations, the full parameter set, and the module API are specified
below.
\end{abstract}

%% ============================================================
\section{Motivation}
\label{sec:motivation}

FX markets contain a large share of non-profit-seeking participants ---
corporate hedgers, central banks managing reserves, and passive
institutional flows --- whose activity creates persistent, exploitable
price inefficiencies. Unlike equities, where the efficient-market
hypothesis has substantial empirical support, the FX market's structural
heterogeneity provides a credible economic mechanism for factor premia.

The dominant structural feature is USD directionality: the first principal
component of cross-sectional FX returns explains approximately 54\% of
variance and loads almost uniformly on all USD pairs. Any systematic
strategy must handle this explicitly. Time-series (directional) strategies
capture USD trends; cross-sectional (relative-value) strategies neutralise
USD exposure via beta constraints against PC1.

We focus on two signals with the strongest empirical support and longest
available histories:

\begin{enumerate}
  \item \textbf{Momentum}: sign-based trend following averaged over
    232 lookback windows (21--252 business days), with hysteresis noise
    control and signal-dispersion deflation.
  \item \textbf{Carry}: the forward--spot spread normalised by return
    volatility, representing an ex-ante Sharpe ratio of the roll yield.
\end{enumerate}

Five additional signals (Value, MSO, COFFEE, CFTC Continuation, CFTC
Reversal) are documented as extensions but not included in the primary
hypothesis test.

%% ============================================================
\section{Mathematical Framework}
\label{sec:math}

\subsection{Symbol Table}

\begin{longtable}{>{\raggedright}p{2.2cm} p{7.5cm} l l}
\toprule
\textbf{Symbol} & \textbf{Meaning} & \textbf{Type} & \textbf{Units} \\
\midrule
\endfirsthead
\toprule
\textbf{Symbol} & \textbf{Meaning} & \textbf{Type} & \textbf{Units} \\
\midrule
\endhead
$r_{t-h,t}$ & Total return (spot + carry) from $t{-}h$ to $t$ & scalar & \% \\
$S_t$ & Raw momentum signal at time $t$ & scalar & -- \\
$\hat{S}_t$ & Preliminary signal (after hysteresis) & scalar & -- \\
$\tilde{\sigma}_t$ & Momentum signal dispersion & scalar & -- \\
$\sigma_{r_i}$ & Volatility of asset $i$ returns & scalar & \% ann. \\
$w_i$ & Portfolio weight for asset $i$ & scalar & fraction \\
$\beta_i$ & Beta of currency $i$ to USD PC1 & scalar & -- \\
$c_i$ & Carry (forward--spot spread) for asset $i$ & scalar & \% \\
$F_i$ & Forward exchange rate for asset $i$ & scalar & USD/FX \\
$S_i$ & Spot exchange rate for asset $i$ & scalar & USD/FX \\
$N$ & Number of currency pairs in pool & int & 24 \\
$h$ & Lookback window & int & bdays \\
$\rho$ & Correlation matrix & matrix & -- \\
$\text{REER}_{i,t}$ & Real effective exchange rate & scalar & index \\
$\text{PROD}_{i,t}$ & Productivity differential & scalar & log ratio \\
$\text{TOT}_{i,t}$ & Terms of trade index & scalar & index \\
$\varepsilon_{i,t}$ & REER misalignment (regression residual) & scalar & log \\
$I^{NC}_{i,t}$ & CFTC non-commercial long positions & scalar & contracts \\
$S^{SNC}_{i,t}$ & CFTC non-commercial short positions & scalar & contracts \\
$\text{MIC}$ & Modified Information Coefficient & scalar & cov \\
$Q$ & Number of factor signals & int & count \\
\bottomrule
\end{longtable}

\subsection{Momentum Signal (Eq.~1--4)}

\paragraph{Eq.~1 --- Momentum Raw Signal.}
\begin{equation}
  S_t = \frac{1}{232} \sum_{h=21}^{252} \operatorname{sign}(r_{t-h,t})
  \label{eq:mom-raw}
\end{equation}
The signal averages the sign of total returns across 232 lookback windows
($h = 21, 22, \ldots, 252$ business days). Bounded: $S_t \in [-1, 1]$.

\paragraph{Eq.~2 --- Momentum Signal Dispersion.}
\begin{equation}
  \tilde{\sigma}_t = \sqrt{
    \frac{1}{232} \sum_{h=21}^{252}
    \bigl(\operatorname{sign}(r_{t-h,t}) - S_t\bigr)^2
  }
  \label{eq:mom-disp}
\end{equation}
Cross-lookback variability of the sign readings. Floored at the 25th
percentile of the cross-asset distribution to prevent division by
near-zero values.

\paragraph{Eq.~3 --- Momentum Hysteresis Noise Control.}
\begin{equation}
  \hat{S}_t = \begin{cases}
    \operatorname{sign}(S_t), & |S_t| \ge \theta \\
    \hat{S}_{t-1},            & |S_t| < \theta
  \end{cases}
  \label{eq:mom-hyst}
\end{equation}
where $\theta = 1/3$. If the raw signal is weak, the prior position is
retained, reducing turnover.

\paragraph{Eq.~4 --- Momentum Final Signal.}
\begin{equation}
  S_t^{\text{final}} = \frac{\hat{S}_t}{
    \max\!\bigl(\tilde{\sigma}_t,\; \tilde{\sigma}^{\text{floor}}_t\bigr)
  }
  \label{eq:mom-final}
\end{equation}
Higher dispersion $\to$ lower conviction $\to$ smaller position.
$\tilde{\sigma}^{\text{floor}}_t$ is the 25th percentile of the
cross-asset $\tilde{\sigma}$ distribution on day $t$.

\subsection{Carry Signal (Eq.~5--7)}

\paragraph{Eq.~5 --- Carry Signal (raw).}
\begin{equation}
  c_i = \frac{S_i - F_i}{F_i}
  \label{eq:carry-raw}
\end{equation}
where $S_i$ = spot, $F_i$ = 1-month forward. Positive carry implies the
long position earns positive roll yield.

\paragraph{Eq.~6 --- Carry Signal (smoothed).}
\begin{equation}
  c_t^{\text{Carry}} = \frac{1}{L} \sum_{h=0}^{L-1} c_{t-h}
  \label{eq:carry-smooth}
\end{equation}
$L$-day moving average for noise reduction.

\paragraph{Eq.~7 --- Carry Signal (risk-adjusted).}
\begin{equation}
  S_{i,t}^{\text{Carry}} = \frac{c_{i,t}^{\text{Carry}}}{\sigma_{r_{i,t}}}
  \label{eq:carry-risk}
\end{equation}
Carry divided by asset return volatility $\sigma_{r_{i,t}}$ (not signal
dispersion). Resembles an ex-ante Sharpe ratio.

\subsection{Value Signal (Eq.~8--9) --- Extension}

\paragraph{Eq.~8 --- DOLS Panel Regression.}
\begin{equation}
  \log(\text{REER}_{i,t}) = \alpha_i
    + \beta^1 \text{PROD}_{i,t}
    + \beta^2 \text{TOT}_{i,t}
    + \theta_0
    + \varepsilon_{i,t}
  \label{eq:value-dols}
\end{equation}
\begin{equation*}
  \theta_0 = \sum_{s=-1}^{1}
    \bigl(\psi_s^1 \Delta\text{PROD}_{i,t+s}
    + \psi_s^2 \Delta\text{TOT}_{i,t+s}\bigr)
\end{equation*}
Dynamic OLS with leads/lags. $\alpha_i$ = country fixed effects.
Estimated per panel (5 panels: EM commodity exporters/importers, East
Asian tigers, G10 commodity exporters/importers).

\paragraph{Eq.~9 --- Value Signal.}
\begin{equation}
  S_{i,t}^V = \varepsilon_{i,t}
  \label{eq:value-signal}
\end{equation}
The residual from Eq.~\ref{eq:value-dols}, converted from REER to USD/FX
space via matrix inversion.

\subsection{Portfolio Construction (Eq.~10--11, 16--17)}

\paragraph{Eq.~10 --- Inverse Volatility Weights (IVW).}
\begin{equation}
  w_i^{\text{IVW}} = \frac{1/\sigma_{r_i}}{\sum_j 1/\sigma_{r_j}}
  \label{eq:ivw}
\end{equation}
Used for time-series momentum and short-term signal portfolios.

\paragraph{Eq.~11 --- Beta Neutralisation.}
\begin{equation}
  \arg\min_w \sum_{i=1}^N (w_i - \tilde{w}_i)^2
  \quad \text{s.t.} \quad \sum_i w_i \beta_i = 0
  \label{eq:beta-neutral}
\end{equation}
$\tilde{w}_i$ = initial weights (equal or rank-based), $\beta_i$ =
rolling 1-year beta to USD PC1.

\paragraph{Eq.~16 --- Max Sharpe Carry Weights.}
\setcounter{equation}{15}
\begin{equation}
  \arg\max_w \frac{\sum_i w_i c_i}{
    \sqrt{\sum_i \sum_j w_i w_j \sigma_{i,j}}
  }
  \quad \text{s.t.} \quad \sum_i w_i \beta_i = 0
  \label{eq:max-sharpe}
\end{equation}
Maximise ex-ante Sharpe of carry portfolio subject to USD-beta neutrality
and boundary constraints.

\paragraph{Eq.~17 --- Time-Series Momentum Final Weights.}
\begin{equation}
  W_{i,t}^{\text{Trend}} = \frac{
    S_{i,t}^{\text{final}} / \sigma_{r_{i,t}}
  }{
    \sum_j \bigl|S_{j,t}^{\text{final}} / \sigma_{r_{j,t}}\bigr|
  }
  \label{eq:ts-weights}
\end{equation}
Combines the momentum signal (already deflated by dispersion via
Eq.~\ref{eq:mom-final}) with inverse volatility weighting. The
denominator $\sigma_{r_{i,t}}$ is asset return volatility, not signal
dispersion --- using signal dispersion here would double-deflate.

\subsection{Short-Term Signals (Eq.~12--15) --- Extensions}

\paragraph{Eq.~12 --- Momentum Spill-Over (MSO).}
The MSO signal captures short-term interest rate momentum spilling into
FX returns:
\begin{enumerate}
  \item $\Delta IR_{i,h}$ = annualised change in 6M interest rate spread
    over $h \in \{21, 42, 63\}$ days.
  \item $\sigma_{IR,h}$ = annualised volatility of daily changes in 6M IR
    differential over same windows.
  \item $\text{ratio}_h = \Delta IR_{i,h} / \sigma_{IR,h}$.
  \item $S_t^{\text{MSO}} = \tfrac{1}{3} \sum_{h} \text{ratio}_h$, then
    1-month moving average.
\end{enumerate}
\textbf{Filter}: Remove asset if 1-year correlation between past 1-day IR
spread moves and future 1-day FX returns is negative.

\paragraph{Eq.~13 --- COFFEE Signal.}
\setcounter{equation}{12}
\begin{equation}
  S_t^{\text{COFFEE}} = \frac{
    \text{NotionalCalls}_{4w} - \text{NotionalPuts}_{4w}
  }{
    \sigma_{1y}(\text{NotionalCalls}_{4w} - \text{NotionalPuts}_{4w})
  }
  \label{eq:coffee}
\end{equation}
European options with $|\delta| \in [0.25, 0.75]$, expiry $< 1$ year.
4-week aggregate volume, standardised by 1-year rolling volatility.
Requires DTCC data (short history, 2014+).

\paragraph{Eq.~14 --- CFTC Continuation Signal.}
\begin{equation}
  S_{i,t}^{\text{CFTC,C}} = \frac{
    \sum_{h=0}^{3} I_{i,t-h}^{NC} - \sum_{h=0}^{3} S_{i,t-h}^{SNC}
  }{
    \sum_{h=0}^{3} I_{i,t-h}^{NC} + \sum_{h=0}^{3} S_{i,t-h}^{SNC}
  }
  \label{eq:cftc-cont}
\end{equation}
4-week summed net positioning ratio from non-commercial traders.

\paragraph{Eq.~15 --- CFTC Reversal Signal.}
\begin{enumerate}
  \item $z_h$ = z-score of net position over lookback
    $h \in \{1\text{M}, 2\text{M}, 3\text{M}\}$.
  \item $\bar{z} = \tfrac{1}{3}(z_{1M} + z_{2M} + z_{3M})$.
  \item $S_i^{\text{CFTC,R}} = -\bar{z} / \sigma_{r_i, 6M}$.
\end{enumerate}
Sign-flipped to represent reversal. Volatility-normalised. Cross-sectional
only (time-series MIC $\approx 0$). Known discretisation risk from weekly
CFTC reporting frequency.

\subsection{Modified Information Coefficient (Eq.~18--19)}

\paragraph{Eq.~18 --- MIC.}
\setcounter{equation}{17}
\begin{equation}
  \text{MIC}_{t,h}^{TS} = \mathbb{E}\!\left[
    \frac{r_{t+h}\, S_t}{\sum_{i=1}^N S_i^2}
  \right]
  \label{eq:mic-ts}
\end{equation}
\begin{equation}
  \text{MIC}_{t,h}^{CS} = \mathbb{E}\!\left[
    \frac{r_{t+h}\, (S_t - \operatorname{median}(S_t))}{
      \sum_{i=1}^N (S_i - \operatorname{median}(S_i))^2
    }
  \right]
  \label{eq:mic-cs}
\end{equation}
Used to evaluate signal predictive power in time-series and
cross-sectional contexts respectively.

%% ============================================================
\section{Pre-Registered Methodology}
\label{sec:methodology}

\subsection{Hypotheses}

\begin{description}
  \item[$H_0$:] FX momentum and carry signals have zero predictive power
    for future currency returns after transaction costs.
  \item[$H_1$:] Multi-horizon momentum and carry signals generate positive
    risk-adjusted returns (Sharpe $> 0.4$) net of transaction costs in
    USD/FX markets.
  \item[Test:] Two-sided $t$-test on annualised Sharpe ratio vs.\ 0.
  \item[$\alpha$:] 0.05.
  \item[Effect size:] 0.4 (minimum Sharpe ratio for economic significance).
\end{description}

\subsection{Data}

\begin{itemize}
  \item \textbf{Universe}: 24 USD/FX pairs --- G10 (AUD, EUR, GBP, CHF,
    SEK, NOK, NZD, CAD, JPY) plus select EM (BRL, CZK, KRW, MXN, PLN,
    RUB, SGD, TRY, TWD, ZAR, HUF, ILS, INR, THB).
  \item \textbf{Frequency}: Daily.
  \item \textbf{Columns}: date, currency\_pair, spot\_rate, forward\_1m,
    forward\_6m (nullable), bid\_ask\_spread, total\_return.
  \item \textbf{Transaction costs}: Fixed bid--ask spread $= 1.5\times$
    long-term historical average.
\end{itemize}

\begin{table}[ht]
\centering
\caption{Data splits}
\label{tab:splits}
\begin{tabular}{llll}
\toprule
\textbf{Split} & \textbf{Start} & \textbf{End} & \textbf{Purpose} \\
\midrule
Train    & 2000-01-01 & 2014-12-31 & Signal calibration \\
Validate & 2015-01-01 & 2018-12-31 & Parameter sensitivity \\
Holdout  & 2019-01-01 & 2024-12-31 & Final evaluation \\
\bottomrule
\end{tabular}
\end{table}

\textbf{Holdout status: untouched.} Results will only be computed on
the holdout split after all pre-registered methodology is locked.

\subsection{Procedure}

\begin{enumerate}
  \item Compute total returns $r_{t-h,t}$ for each currency from spot and
    1M forward data.
  \item \textbf{Momentum signal}: For each currency on each day, compute
    the raw signal (Eq.~\ref{eq:mom-raw}), dispersion
    (Eq.~\ref{eq:mom-disp}), apply hysteresis (Eq.~\ref{eq:mom-hyst}),
    and deflate (Eq.~\ref{eq:mom-final}).
  \item \textbf{Carry signal}: Compute raw carry (Eq.~\ref{eq:carry-raw}),
    smooth (Eq.~\ref{eq:carry-smooth}), risk-adjust
    (Eq.~\ref{eq:carry-risk}).
  \item \textbf{Time-series portfolio}: Apply inverse volatility weighting
    (Eq.~\ref{eq:ivw}) to momentum signals. Normalise absolute weight sum
    to 1. Apply position caps at 15\%.
  \item \textbf{Cross-sectional portfolio}: Rank currencies by carry
    signal, long top half, short bottom half. Beta-neutralise to USD PC1
    (Eq.~\ref{eq:beta-neutral}). Optionally use max-Sharpe weights
    (Eq.~\ref{eq:max-sharpe}).
  \item \textbf{Tranching}: Split each portfolio into 20 daily tranches
    for momentum/carry (monthly rebalance), 5 tranches for short-term
    signals (weekly rebalance).
  \item \textbf{Backtest}: Execute with T+1 lag. Compute gross and net
    returns, turnover, cumulative PnL.
  \item \textbf{Evaluation}: Compute Sharpe, CAGR, max drawdown, Calmar,
    Sortino, average turnover. Apply $t$-test to Sharpe ratios.
\end{enumerate}

%% ============================================================
\section{Expected Results}
\label{sec:expected}

Under $H_1$, we expect the following performance characteristics on
the train and validation splits:

\begin{table}[ht]
\centering
\caption{Pre-registered success criteria}
\label{tab:criteria}
\begin{tabular}{llll}
\toprule
\textbf{Metric} & \textbf{Threshold} & \textbf{Direction} & \textbf{Rationale} \\
\midrule
Sharpe (momentum TS) & 0.4 & $>$ & Paper reports 0.60 \\
Sharpe (carry CS)     & 0.5 & $>$ & Paper reports 0.79 \\
Max drawdown          & 25\% & $<$ & Risk constraint \\
Avg daily turnover    & 10\% & $<$ & Cost constraint \\
\bottomrule
\end{tabular}
\end{table}

The thresholds are set below reported values to account for
implementation differences, data source discrepancies, and the
inherent optimism in reported backtests. If train-split Sharpe falls
below 0.3 or is sign-inconsistent with the paper, we investigate
data/implementation before proceeding.

%% ============================================================
\section{Implementation Notes}
\label{sec:implementation}

\subsection{Module Architecture}

The implementation comprises four modules with ten public functions:

\begin{table}[ht]
\centering
\caption{Module API summary}
\label{tab:modules}
\begin{tabular}{lll}
\toprule
\textbf{Module} & \textbf{Function} & \textbf{Returns} \\
\midrule
\texttt{signals}   & \texttt{compute\_momentum\_signal} & DataFrame \\
                    & \texttt{compute\_carry\_signal}    & Series \\
                    & \texttt{compute\_mso\_signal}      & Series \\
\midrule
\texttt{portfolio}  & \texttt{build\_ts\_weights}   & DataFrame \\
                    & \texttt{build\_cs\_weights}   & DataFrame \\
                    & \texttt{apply\_tranching}     & DataFrame \\
\midrule
\texttt{risk}      & \texttt{estimate\_covariance} & ndarray \\
                    & \texttt{compute\_usd\_beta}   & Series \\
\midrule
\texttt{backtest}  & \texttt{run\_backtest}        & DataFrame \\
                    & \texttt{compute\_metrics}     & dict \\
\bottomrule
\end{tabular}
\end{table}

\subsection{Parameter Table}

\begin{table}[ht]
\centering
\caption{Strategy parameters}
\label{tab:params}
\begin{tabular}{llll}
\toprule
\textbf{Parameter} & \textbf{Value} & \textbf{Tunable} & \textbf{Description} \\
\midrule
lookback\_min & 21 & No & Min lookback window (bdays) \\
lookback\_max & 252 & No & Max lookback window (bdays) \\
hysteresis\_threshold & 0.333 & Yes & Noise control threshold \\
vol\_decay\_diagonal & 252 & Yes & EWMA decay for variances \\
vol\_decay\_offdiag & 756 & Yes & EWMA decay for correlations \\
rebalance\_freq\_medium & 20 & No & Monthly rebalance (bdays) \\
rebalance\_freq\_short & 5 & No & Weekly rebalance (bdays) \\
max\_position\_pct & 0.15 & Yes & Per-asset position cap \\
dispersion\_floor\_pctl & 0.25 & Yes & Floor for $\tilde{\sigma}$ \\
n\_currencies & 24 & No & Universe size \\
\bottomrule
\end{tabular}
\end{table}

%% ============================================================
\section{Risk Analysis}
\label{sec:risk}

\begin{description}
  \item[Carry crash.] EM carry trades are exposed to risk-off events
    (e.g., Aug 2015, Mar 2020). Beta neutralisation mitigates but does
    not eliminate this. No explicit drawdown stop is specified.

  \item[Trend reversal regimes.] Post-2008 QE compressed FX volatility
    and weakened momentum signals. The paper acknowledges declining carry
    contribution since 2010. Regime-conditional analysis is warranted.

  \item[Value signal overfitting.] DOLS panel regressions on macro data
    (GDP, ToT) are subject to data revision, look-ahead bias, and
    parameter instability. The 5-panel structure adds degrees of
    freedom. Value is designated as an extension, not a primary signal.

  \item[COFFEE data dependency.] DTCC data has short history (2014+) and
    limited accessibility. MIC levels may be overstated due to
    survivorship in the available sample.

  \item[CFTC discretisation risk.] Weekly CFTC snapshots introduce
    material discretisation error, especially for the reversal signal.
    The paper explicitly tests and warns about this.

  \item[Correlation regime shifts.] IVW vs.\ MCW choice assumes average
    cross-asset correlation stays below ${\sim}50\%$. If FX correlations
    spike (e.g., USD crisis), IVW underperforms minimum-correlation
    weighting.

  \item[Transaction cost sensitivity.] Some short-term signals show
    Sharpe degradation after costs (MSO: 0.33 before costs, lower
    after). The fixed $1.5\times$ spread assumption is a simplification;
    real costs are time-varying and liquidity-dependent.
\end{description}

%% ============================================================
% \section{Results}
% \label{sec:results}
%
% UNCOMMENT ONLY AFTER HOLDOUT EVALUATION.
%
% Present: equity curves, Sharpe ratios, drawdown analysis, turnover,
% t-test results, comparison to paper-reported values.
%
% \begin{table}[h]
% \centering
% \caption{Backtest results (holdout period)}
% \label{tab:results}
% \begin{tabular}{lrrr}
% \toprule
% \textbf{Metric} & \textbf{Momentum TS} & \textbf{Carry CS} & \textbf{Combined} \\
% \midrule
% Sharpe & -- & -- & -- \\
% CAGR (\%) & -- & -- & -- \\
% Max DD (\%) & -- & -- & -- \\
% Calmar & -- & -- & -- \\
% Sortino & -- & -- & -- \\
% Avg turnover (\%) & -- & -- & -- \\
% $t$-stat (Sharpe $\neq 0$) & -- & -- & -- \\
% $p$-value & -- & -- & -- \\
% \bottomrule
% \end{tabular}
% \end{table}

%% ============================================================
\bibliographystyle{plain}
\bibliography{refs}

\end{document}
